\documentclass{kththesis}

\usepackage{blindtext} % This is just to get some nonsense text in this template, can be safely removed

\usepackage{csquotes} % Recommended by biblatex
\usepackage{biblatex}
\addbibresource{references.bib} % The file containing our references, in BibTeX format


\title{This is the English title}
\alttitle{Detta är den svenska översättningen av titeln}
\author{Osquar Student}
\email{osquar@kth.se}
\supervisor{Lotta Larsson}
\examiner{Lennart Bladgren}
\programme{Master in Computer Science}
\school{School of Computer Science and Communication}
\date{\today}


\begin{document}

% Frontmatter includes the titlepage, abstracts and table-of-contents
\frontmatter

\titlepage

\begin{abstract}
  English abstract goes here.

  \blindtext
\end{abstract}


\begin{otherlanguage}{swedish}
  \begin{abstract}
    Träutensilierna i ett tryckeri äro ingalunda en oviktig faktor,
    för trevnadens, ordningens och ekonomiens upprätthållande, och
    dock är det icke sällan som sorgliga erfarenheter göras på grund
    af det oförstånd med hvilket kaster, formbräden och regaler
    tillverkas och försäljas Kaster som äro dåligt hopkomna och af
    otillräckligt.
  \end{abstract}
\end{otherlanguage}


\tableofcontents


% Mainmatter is where the actual contents of the thesis goes
\mainmatter


\chapter{Introduction}

We use the \emph{biblatex} package to handle our references.  We therefore use the
command \texttt{parencite} to get a reference in parenthesis, like this
\parencite{Erdweg2013}.  It is also possible to include the author
as part of the sentence using \texttt{textcite}, like talking about
the work of \textcite{Erdweg2013}.

\Blindtext

\section{Research Question}

\blindtext

\chapter{Methods}

\blindtext

\chapter{Related Work}

The following section currently contains short, basic descriptions of the capabilities of some various systems/methods. Most of it is orientational, no real deep study of any of them has been done. The descriptions are also poor.

\section{SDF}

\textcite{Heering1989} introduce the syntax definition formalism SDF, a method for declaratively defining syntax, including lexical syntax, context free syntax, and various disambiguation methods.

\section{Layout-sensitive SDF}

\textcite{Erdweg2013} extend SDF to handle layout-sensitive grammars in a declarative manner through annotations in each production restricting their relative positions. To achieve this they determine a limited subset of possible layout restrictions that are sufficient to describe the rules of most layout-sensitive languages.

\section{$\lambda_m$}

$\lambda_m$ \cite{Herman2010} is an extension of lambda calculus with macros and macro type signatures. These signatures describe the structure of macro arguments and results, including binding structure, and form an inspirational basis to the approach taken by this project in regard to hygiene and name binding.

Some notable differences exist however:
\begin{itemize}
  \item $\lambda_m$ macros can only affect bindings within themselves, a natural consequence of lambda calculus lacking constructions where bindings are not nested, e.g., a local variable in C can be used for the remainder of the scope, not only in some sub-expression, which is the case for a let-expression. This project handles such binding structures, which imposes some additional restrictions on what the implementation of a macro can do.
  \item The macro type signatures in this project are never fully explicit, they appear spread out through each macro definition amongst other aspects of the definition. $\lambda_m$ in contrast uses a complete type signature for each macro.
\end{itemize}

$\lambda_m$ also introduces definitions of $\alpha$-equivalence and hygiene that do not depend on macro expansion and also place focus on allowing reasoning about unexpanded programs.

\section{Romeo}

\textcite{Stansifer2014} introduce Romeo, a language extending $\lambda_m$ by providing full syntax manipulation capabilities while still retaining hygiene. $\lambda_m$ is limited to pattern matching and replacing, while Romeo can manipulate terms as data.

\section{Copper and Silver}

Copper \cite{VanWyk2007} defines a grammar and a lexical scanner that works in tandem, where the scanner only returns tokens that the grammar defines as valid next tokens given the current state of the parse. It uses a modified LR parsing algorithm. The combination manages to parse more languages, since the scanner can work unambiguously in more cases. Silver \cite{VanWyk2010} uses attribute grammars to specify modular language extensions on some host language, along with guarantees on some of their behaviour under composition \cite{Kaminski2017}. Silver uses Copper for grammar specification.

\section{SoundX}

\textcite{Lorenzen2016} introduce SoundX, a system for specifying syntax and type rules for a base language, and then language extensions through rewritings (macros). Each rewriting also has specified type rules, detailing what is required for a successful type derivation. Through these it can be statically checked that assuming there is a derivation of the original code, then there will be a derivation in the base language of the rewriting, i.e. it can be statically checked that the rewriting is correct as written, and each use can be checked for correctness without actually performing the rewriting. SDF is used for the syntactic specification of languages and language extensions.

SoundX can technically fail during rewriting. Rewriting is done on derivation trees, and the same construct may appear in multiple premises. If those appearances rewrite into different, incompatible, desugarings the process will fail.

\printbibliography[heading=bibintoc] % Print the bibliography (and make it appear in the table of contents)

\appendix

\chapter{Unnecessary Appended Material}

\end{document}
